\documentclass[a4paper,twoside,12pt]{article}

\usepackage{xltxtra}

%% STRUCTURE

\usepackage{xspace}
\usepackage{expl3}
%\usepackage{enumitem}
\usepackage[olditem]{paralist}
\usepackage{todo}

\usepackage{amsmath,amssymb,amsthm,thmtools}
\usepackage{braket}

\usepackage[nounderscore]{syntax}

\usepackage{bussproofs}
\usepackage{graphicx}
\usepackage{stmaryrd}
\usepackage{bbold}

\usepackage{tikz}
\usepackage{tikz-cd}

\declaretheorem[name=Teorema]{theorem}
\declaretheorem[name=Lema]{lemma}
\declaretheorem[name=Corolario]{corollary}
\declaretheorem[name=Definición]{definition}
\declaretheorem[name=Proposición]{proposition}
\declaretheorem[name=Ejemplo]{example}
\declaretheorem[name=Ejercicio]{exercise}
\declaretheorem[name=Excursus]{excursus}
\declaretheorem[name=Preguntas]{questions}

\usepackage{biblatex}
\addbibresource[location=remote]{http://www.citeulike.org/bibtex/user/miguelpagano/library?fieldmap=posted-at:posted-date&clean_urls=0}
\title{Semántica para While}

\newcommand{\trees}{\mathbb{D}}

%% Elementos de D
\newcommand{\nil}{\mathsf{nil}}
\newcommand{\cons}[2]{(#1\, .\, #2)}
\newcommand{\falseD}{\mathit{false}}
\newcommand{\trueD}{\mathit{true}}

%% Expresiones
\newcommand{\vars}{\mathcal{V}}
\newcommand{\hd}{\mathsf{hd}}
\newcommand{\tl}{\mathsf{tl}}
\newcommand{\conse}{\mathsf{cons}}
\newcommand{\isEq}{\mathop{?\!\!=}}

%% Comandos
\newcommand{\assign}{\mathop{:=}}
\newcommand{\seq}{\mathop{;}}
\newcommand{\whilec}{\mathsf{while}}
\newcommand{\doc}{\mathsf{do}}
\newcommand{\whilestmt}[2]{\whilec\,#1\,\doc\,#2}
\newcommand{\seqstmt}[2]{#1\,\seqc\,#2}
\newcommand{\ifstmt}[3]{\mathsf{if}\,#1\,\mathsf{then}\,#2\,\mathsf{else}\,#3}

%% Programas
\newcommand{\readV}{\mathsf{read}}
\newcommand{\writeV}{\mathsf{write}}

\newcommand{\sem}[2]{\llbracket #1 \rrbracket #2}
\newcommand{\semR}[3]{\llbracket #2 \rrbracket^{#1} #3}
\newcommand{\upd}[3]{[#1 \mid #2 \colon #3]}
\newcommand{\semf}[1]{\llbracket #1 \rrbracket}
\newcommand{\liftFun}[1]{{#1}_{\bot\!\!\bot}}
\newcommand{\fixpoint}[2]{\mathbf{Y}_{#1}\,#2}
\newcommand{\getVar}[1]{\mathop{get}\, #1}

\setlength{\grammarparsep}{2pt}

% Semántica operacional
\newcommand{\bigstep}{\Rightarrow}


\begin{document}

El libro ``Computability and Complexity'' de Neil
Jones %\cite{jones-comp}
presenta un lenguaje cuyos datos permiten manipular programas escritos
en el mismo lenguaje. La intención de este documento es dar una
semántica denotacional del lenguaje. En realidad en el libro se
presentan varios lenguajes, nuestra exposición intentará ser gradual
de la misma manera que el libro.

\section{El lenguaje While}
\label{sec:syntax}

La sintaxis del lenguaje se define a partir de árboles binarios
sobre un único átomo:

\begin{definition}[Árboles]
  El conjunto $\trees$ de \emph{árboles} es el menor conjunto que satisface:
  \begin{enumerate}
  \item $\nil \in \trees$ 
  \item Si $d_1 \in \trees$ y $d_2 \in \trees$, entonces
    $\cons{d_1}{d_2}\in\trees$.
  \end{enumerate}
\end{definition}

Los elementos de $\trees$ serán los ``datos'' que manipularán los
programas While; para ello definimos una sintaxis para las expresiones:

\begin{definition}[Expresiones]
  Dado un conjunto numerable $\vars$, representado por $\synt{var}$ en
  la grámatica, de variables la sintaxis abstracta de las expresiones
  está dada por:
  \begin{grammar}
  <atom> ::= $\nil$

  <exp> ::=  <var> | <atom> | $\hd$ <exp> | $\tl$ <exp> | $\conse$ <exp> <exp> | <exp> $\isEq$ <exp>

  <comm> ::= <var> $\assign$ <exp> 
          | <comm> $\seq$ <comm> 
          | $\whilec$ <exp> $\doc$ <comm> 

  <prog> ::= $\readV$ <var> $\seq$ <comm> $\seq$ $\writeV$ <var>
\end{grammar}
\end{definition}

La semántica denotacional de expresiones toma como parámetro un estado
que mapea variables a elementos de $\trees$. Para representar los
booleanos con elementos de $\trees$ alcanza con definir dos valores
distintos: a falso lo representamos con $\falseD = \nil$, mientras que
a verdadero con $\trueD = \cons \nil \nil$.


\begin{definition}[Semántica]
  Sea $\Sigma = \vars \to \trees$ el conjunto de \emph{estados}; sea
  $\sigma \in \Sigma$. Definimos por recursión la semántica de
  expresiones:
  \[
  \array{r@{\,=\,}l@{\quad}l@{}}%
  \sem{v}{\sigma} & \sigma\,v & {} \\
  \sem{\nil}{\sigma} & \nil & {} \\
  \sem{\conse\,e\,e'}{\sigma} & \cons{\sem e \sigma}{\sem{e'}{\sigma}} & {} \\
  \sem{\hd\,e}{\sigma} & d & \text{ si } \sem{e}{\sigma} = \cons{d}{d'} \\
  \sem{\hd\,e}{\sigma} & \nil & \text{ caso contrario }\\
  \sem{\tl\,e}{\sigma} & d' & \text{ si } \sem{e}{\sigma} = \cons{d}{d'} \\
  \sem{\tl\,e}{\sigma} & \nil & \text{ caso contrario }\\
  \sem{e\,\isEq\,e'}{\sigma} & \trueD & \text{ si } \sem{e}{\sigma} = \sem{e'}{\sigma} \\
  \sem{e\,\isEq\,e'}{\sigma} & \falseD & \text{ caso contrario } \\
  \endarray
  \]
\end{definition}

Uno esperaría que la semántica de comandos sea una función que toma un
estado y devuelve otro estado; el comando de repetición complica las
cosas y requiere que nos movamos a funciones parciales. En el libro la
semántica de un comando está dada en términos de una semántica
operacional, de este modo, un comando es indefinido si no existe una
derivación (finita) construida utilizando las reglas de la semántica
operacional.

Otro camino, quizás más matemático, es recurrir a resultados de teoría
de dominios y pensar que la semántica de comandos sean funciones
continuas de $\Sigma$ en $\Sigma_\bot$. Recordemos que $\Sigma_\bot$
es el dominio \emph{llano} obtenido al extender $\Sigma$ con un nuevo
elemento $\bot$ y definiendo el siguiente orden: $x \leqslant y$ si y
sólo si $x = \bot$ ó $x = y$. Puesto que $\Sigma_\bot$ es un dominio, entonces
$\Sigma\to \Sigma_\bot$ también es un dominio; el orden entre
funciones está dado por el orden punto-a-punto: $f \leqslant g $ si y
sólo si ``para todo $\sigma$, $f\,\sigma \leqslant g\,\sigma$''. 

Un teorema de teoría de dominios indica cómo encontrar el menor punto
fijo de una endo-función continua. Sea $f \colon D \to D$, donde $D$
es un dominio, entonces podemos construir la siguiente cadena 
$f\,\bot_D \leqslant f\,(f\,\bot_D) \leqslant f^3\,\bot_D\leqslant
\ldots$.
Puesto que $D$ es un dominio, entonces sabemos que existe el supremo
de esa cadena y es fácil ver que ese supremo es un punto fijo para
$f$. Dada una función $f \colon D \to D$, utilizamos $\fixpoint{D}{f}$
para referirnos al menor punto fijo dado por esa construcción.

\begin{definition}[Semántica de comandos]
  \[
  \array{r@{\,=\,}l@{\quad}l@{}}%
  \sem{v \assign e}{\sigma} & \upd{\sigma}{v}{\sem e \sigma} & {} \\
  \sem{c \seq c'}{\sigma} & \liftFun{\semf{c'}}\,(\sem c \sigma) & {} \\
  \sem{\whilec\, e\, \doc\, c}{\sigma} & \fixpoint{\Sigma_\bot \to \Sigma_\bot}{F}\,\sigma & \text{ donde } \\
  \endarray
  \]

  \[
  \array{r@{\,}l}%
  F \colon & (\Sigma \to \Sigma_\bot) \to (\Sigma \to \Sigma_\bot) \\
  F\,g\,\sigma = &
  \begin{cases}
    \sigma & \text{ si } \sem{e}{\sigma} = \falseD \\
    \liftFun{g}\,(\sem c \sigma) & \text{ caso contrario } 
  \end{cases}
  \endarray
  \]
\end{definition}

Dada la semántica de comandos es fácil definir la de programas, tomamos
un elemento $d \in D$ y un estado $\sigma$, actualizamos el valor de la
variable de entrada en $\sigma$ con $d$, calculamos la semántica del comando
y el resultado será o bien $\bot$ si el programa no termina o bien el valor
de la variable de salida en el estado alcanzado.
\begin{definition}[Semántica de programas]
  \[
  \semf{\readV\, v\seq\, c\,\seq \writeV\, w}\, d\, \sigma = \liftFun{(\getVar{w})}\,(\sem{c}{\upd{\sigma}{v}{d}})
  \] 
\end{definition}

\section{El lenguaje reflexivo}
\label{sec:refl}

En esta sección veremos cómo dar semántica al lenguaje extendido
con la estrella, que hace referencia al propio texto del programa,
y con la expresión $\mathsf{univ}$. La única sintaxis que cambia
es la de expresiones:
\newcommand{\univ}{\mathsf{univ}}

\begin{definition}[Expresiones para el lenguaje reflexivo]\hfill

  \begin{grammar}
    <exp> ::=  $\ldots$ | $\star$ | $\univ$ <exp> <exp>
  \end{grammar}
\end{definition}

Expliquemos la semántica informal de estas nuevas expresiones: la
semántica de $\star$ es el código del programa mismo codificado como
un árbol. Por otro lado para evaluar $\univ\,e_1\,e_2$ debemos primero
evaluar $e_1$ obteniendo el valor $d_1$: si $d_1$ resulta ser la
codificación de un programa $p$ (por ejemplo si $e_1$ es $\star$ entonces
sabremos efectivamente que $d_1$ será la codificación de un programa),
luego evaluamos $e_2$ obteniendo el valor, digamos, $d_2$; finalmente
evaluamos $p$ en el estado actual con valor inicial $d_2$. 

\paragraph{Codificando While como árboles}

\newcommand{\toD}[1]{\lfloor #1 \rfloor }
\newcommand{\toDF}{\toD{\_}}
\newcommand{\fromD}[1]{\lceil #1 \rceil }
\newcommand{\fromDF}{\fromD{\_}}
A los fines prácticos vamos a tomar $\vars = \mathbb N$, esto no produce
ninguna dificultad más que de legibilidad de programas. Es claro que podemos
codificar los naturales como elementos de $\trees$:
\[
\begin{array}{r@{\,}l}
\toDF \colon & \mathbb N \to \trees\\
\toD{0} & = \nil\\
\toD{n+1} & = \cons{\nil}{\toD n}
\end{array}
\]
Es decir, los numerales en $\trees$ se codifican como listas con $n+1$
$\nil$s.

Para codificar expresiones y comandos lo que haremos es construir
listas de dos elementos $\cons \gamma \iota$: el primer elemento es
fijo $\gamma = \cons \nil \nil$ y nos indica que el segundo elemento
debemos interpretarlo como un constructor; para esto enumeramos los
constructores:

\begin{minipage}[t]{0.5\linewidth}
  \[
  \begin{array}{r@{\,=\,}l}
    \mathsf{var} & \toD 1\\
    \mathsf{atom} & \toD 2\\
    \mathsf{hd} & \toD 3\\
    \mathsf{tl} & \toD 4\\
    \mathsf{cons} & \toD 5\\
    \mathsf{eq} & \toD 6\\
  \end{array}
  \]
\end{minipage}
\begin{minipage}[t]{0.5\linewidth}
\[
\begin{array}{r@{\,=\,}l}
  \mathsf{assgn} & \toD 7\\
  \mathsf{seq} & \toD 8\\
  \mathsf{while} & \toD 9\\
  \mathsf{star} & \toD{10}\\
  \mathsf{univ} & \toD{11}\\
\end{array}
\]
\end{minipage}

Ahora para codificar una expresión nos construimos una lista que tiene
primero el constructor y luego tantos elementos más como argumentos
tome ese constructor; para aligerar la carga nos referimos con $\mathsf{c}$ al siguiente elemento de $D$: $\cons{\gamma}{\mathsf{c}}$.
\[
\begin{array}{r@{\,=\,}l@{\quad}l}
  \toD{v} & \cons{\mathsf{var}}{\toD v} & \text{ donde $v$ es una variable} \\
  \toD{\nil} & \cons{\mathsf{atom}}{\nil} & \\
  \toD{\conse\,e\,e'} & \cons{\mathsf{cons}}{\cons{\toD{e}}{\toD{e'}}} & \\
  \toD{\hd\,e} & \cons{\mathsf{hd}}{\toD{e}} & \\
  \toD{\tl\,e} & \cons{\mathsf{tl}}{\toD{e}} & \\
  \toD{e\,\isEq\,e'} & \cons{\mathsf{eq}}{\cons{\toD{e}}{\toD{e'}}} & \\
  \toD{\star} & \mathsf{star} & \\
  \toD{\univ\,e\,e'} & \cons{\mathsf{univ}}{\cons{\toD{e}}{\toD{e'}}} & 
\end{array}
\]
Para los comandos no hay más que usar los otros ``constructores'':
\[
\begin{array}{r@{\,=\,}l}
  \toD{v\,\assign\, e} & \cons{\mathsf{assgn}}{\cons{\toD v}{\toD e}} \\
  \toD{c\seq c'} & \cons{\mathsf{seq}}{\cons{\toD{c}}{\toD{c'}}} \\
  \toD{\whilec\,e\,\doc\,c} & \cons{\mathsf{while}}{\cons{\toD{e}}{\toD c}} 
\end{array}
\]
Finalmente los programas los codificamos como una lista con tres elementos:
\[\toD{\readV\,v\,\seq\, c\,\seq \writeV\, w} = 
  \cons{\toD v}{\cons{\toD c}{\toD w}}
\]
Resaltemos que con $\toD v$ nos referimos a
$\cons{\mathsf{var}}{\toD v}$, y análogamente con $\toD w$.

No es díficil, pero es tedioso, dar funciones parciales $\fromDF$ que
\emph{proyecten} árboles en naturales, expresiones, comandos y
programas, de tal manera que $\fromD{\toD{n}}=n$ para todo natural $n$
(y análogamente para las otras entidades sintácticas). Puesto que
$\toDF$ no es sobre, no podemos esperar que $\fromDF$ sea total y
satisfaga $\toD{\fromD d}=d$.

\paragraph{Semántica formal}

Dada la descripción informal de la semántica de $\univ$ debería ser
claro que ahora la semántica de las expresiones no puede ser más una
función con tipo $\synt{exp} \to \Sigma \to \trees$, sino que el tipo
será $\synt{exp} \to \Sigma \times \trees_{\bot}$. A su vez la
inclusión de $\star$ requiere que tomemos un programa (el programa en
el cual ocurre la expresión $\star$) como argumento.
Otra variación notable respecto al lenguaje considerado anteriormente
es que ahora la semántica de expresiones, comandos y programas debe ser
dada simultáneamente. 

\newcommand{\inj}[1]{\iota_\uparrow\,#1} Introduzcamos algunas
funciones que nos faciliten la notación en las ecuaciones semánticas.
En este apartado utilizamos las siguientes convenciones: $d,d'\in\trees$,
$a,b\in\trees_\bot$
\begin{description}
\item[inyección] $\inj d \in\trees_{\bot}$
\item[concatenación] 
  $a \centerdot b =
  \begin{cases}
    \bot & \text{ si } a = \bot \text{ ó } b = \bot\\
    \inj{\cons{d}{d'}} & \text{ si } a = \inj{d} \text{ y } b = \inj{d'}
  \end{cases}
  $
\item[proyecciones] 
  \[
  \begin{array}{r@{\,}l}
  \pi_i \colon& \trees_\bot \to \trees \\
  \pi_i a =&
  \begin{cases}
   \bot & \text{ si } a = \bot\\
   \inj{d_i} & \text{ si } a = \inj{\cons{d_1}{d_2}}\\
   \inj{\nil} & \text{ si } a = \nil 
  \end{cases}
  \end{array}
  \]
\item[igualdad] 
  $a \overset{?}{=} b =
  \begin{cases}
    \bot & \text{ si } a = \bot \text{ ó } b = \bot\\
    \inj{\trueD} & \text{ si } a = \inj{d} = b\\
    \inj{\falseD} & \text{ caso contrario}
  \end{cases}
  $
\item[actualización]
$\mathop{upd}\,\sigma\,v\,a =
\begin{cases}
  \bot & \text{ si } a = \bot \\
  [\sigma \mid v : d ] & \text{ si } a = \inj{d}
\end{cases}
$
\item[condicional]
  $\mathop{cond}\,a\,x\,y =
  \begin{cases}
    \bot & \text{ si } a = \bot\\
    x & \text{ si } a = \inj{\falseD}\\
    y & \text{ si } a = \inj{d} \text{ con } d\neq \falseD
  \end{cases}
  $
\end{description}

Como ya dijimos, dado un elemento $d\in\trees$, $\fromD d$ puede o no
estar definida; podemos ``internalizar'' la parcialidad cambiando el
tipo de $\fromDF \colon D \to \synt{prg}$ a
$\fromDF \colon D \to \synt{prg}_\bot$. 

\begin{definition}[Semántica]
  Sea $\Sigma = \vars \to \trees$ el conjunto de \emph{estados}; sea
  $\sigma \in \Sigma$. Definimos simultáneamente  la semántica de
  expresiones, comandos y programas:
  \[
  \array{r@{\;=\;}l}%
  \semR{p}{v}{\sigma} & \inj{(\sigma\,v)} \\
  \semR{p}{\nil}{\sigma} & \inj{\nil} \\
  \semR{p}{\conse\,e\,e'}{\sigma} & \semR{p}{e}{\sigma}\centerdot \semR{p}{e'}{\sigma} \\
  \semR{p}{\hd\,e}{\sigma} & \pi_1\, (\semR{p}{e}{\sigma}) \\
  \semR{p}{\tl\,e}{\sigma} & \pi_2\, (\semR{p}{e}{\sigma}) \\
  \semR{p}{e\,\isEq\,e'}{\sigma} & \semR{p}{e}{\sigma} \overset{?}{=} \semR{p}{e'}{\sigma} \\
  \semR{p}{\star}{\sigma} & \inj{\toD{p}} \\
  \semR{p}{\univ\,e\,e'}{\sigma} & 
  \liftFun{(\liftFun{(\liftFun{\sem{\_}{ }} \circ \fromDF)}\, (\semR{p}{e}{\sigma}))}\, (\semR{p}{e'}{\sigma})\,\sigma \\  
  \endarray
  \]
  La semántica de comandos no cambia, sólo llevamos como parámetro
  el programa para dar la semántica de expresiones:
  \[
  \array{r@{\;=\;}l@{\quad}l}%
  \semR{p}{v\,\assign\,e}{\sigma} & \mathop{upd}\,\sigma\,v\,\semR{p}{e}{\sigma} \\
  \semR{p}{c\,\seq\, c'}{\sigma} & \liftFun{(\semR{p}{c'})}\,(\semR{p}{c}{\sigma})\\
  \semR{p}{\whilec\, e\, \doc\, c}{\sigma} & \fixpoint{\Sigma_\bot \to \Sigma_\bot}{F}\,\sigma & \text{ donde } \\
  \endarray
  \]
  \[
  \array{r@{\,}l}%
  F \colon & (\Sigma \to \Sigma_\bot) \to (\Sigma \to \Sigma_\bot) \\
  F\,g\,\sigma =& \mathop{cond}\, (\semR{p}{e}{\sigma})\, \sigma\, (\liftFun{g}\,(\semR p c \sigma))
  \endarray
  \]
  Finalmente la semántica de un programa es:
  \[\semf{\readV\,v\,\seq\,c\,\seq\,\writeV\,w}\,d\,\sigma = 
  \liftFun{(\mathop{get}\,w)}\,(\semR{p}{c}{[\sigma\mid v:d]}) \quad\text{ donde }\]
  \[ p = \readV\,v\,\seq\,c\,\seq\,\writeV\,w\]
  \end{definition}

  Así como la acabamos de definir la semántica del lenguaje reflexivo NO es
  dirigida por sintaxis; para corregir este defecto conceptual podemos hacer
  la siguiente modificación:
  \[\semf{\readV\,v\,\seq\,c\,\seq\,\writeV\,w}\,d\,\sigma = 
  \liftFun{(\mathop{get}\,w)}\,(\semR{d'}{c}{[\sigma\mid v:d]}) \quad\text{ donde }\]
  \[ d' =   \cons{\toD v}{\cons{\toD c}{\toD w}}\]
  Es decir ahora la semántica de comandos (y la de expresiones) no toma como
  parámetro un programa, sino un elemento de $\trees$. La otra modificación
  respecto a la semántica de la definición anterior es en la ecuación para $\star$:
  \[ \semR{d}{\star}{\sigma}\,=\,\inj{d}  \]

\section{Semántica operacional}

En esta sección damos la semántica operacional big-step de While. Para
ello damos tres conjuntos definidos mutuamente de reglas $\bigstep_E$,
$\bigstep_C$ y $\bigstep_P$ de evaluación de expresiones, comandos y
programas respectivamente. Las configuraciones no terminales para
expresiones serán triplas $(\sigma,p,e)$ y las configuraciones
terminales son valores $d \in \trees$.

\begin{minipage}{.5\linewidth}
\begin{prooftree}
  \AxiomC{ }
  \RightLabel{$(\mathit{nil})$}
  \UnaryInfC{$(\sigma,p,\nil) \bigstep_E \nil$}
\end{prooftree}
\end{minipage}
\begin{minipage}{.5\linewidth}
\begin{prooftree}
  \AxiomC{ }
  \RightLabel{$(\mathit{var})$}
  \UnaryInfC{$(\sigma,p,v) \bigstep_E \sigma\,v$}
\end{prooftree}
\end{minipage}

\begin{prooftree}
  \AxiomC{$(\sigma,p,e) \bigstep_E d$}
  \AxiomC{$(\sigma,p,e') \bigstep_E d'$}
  \RightLabel{$(\mathit{cons})$}
  \BinaryInfC{$(\sigma,p,\conse\,e\,e') \bigstep_E \cons{d}{d'}$}
\end{prooftree}

\begin{minipage}{.5\linewidth}
  \begin{prooftree}
    \AxiomC{$(\sigma,p,e) \bigstep_E \cons{d}{d'}$}
    \RightLabel{$(\mathit{hd})$}
    \UnaryInfC{$(\sigma,p,\hd\,e) \bigstep_E d$}
  \end{prooftree}
\end{minipage}
\begin{minipage}{.5\linewidth}
  \begin{prooftree}
    \AxiomC{$(\sigma,p,e) \bigstep_E \nil$}
    \RightLabel{$(\mathit{hd-nil})$}
    \UnaryInfC{$(\sigma,p,\hd\,e) \bigstep_E \nil$}
  \end{prooftree}
\end{minipage}

\begin{minipage}{.5\linewidth}
  \begin{prooftree}
    \AxiomC{$(\sigma,p,e) \bigstep_E \cons{d}{d'}$}
    \RightLabel{$(\mathit{tl})$}
    \UnaryInfC{$(\sigma,p,\tl\,e) \bigstep_E d'$}
  \end{prooftree}
\end{minipage}
\begin{minipage}{.5\linewidth}
  \begin{prooftree}
    \AxiomC{$(\sigma,p,e) \bigstep_E \nil$}
    \RightLabel{$(\mathit{tl-nil})$}
    \UnaryInfC{$(\sigma,p,\tl\,e) \bigstep_E \nil$}
  \end{prooftree}
\end{minipage}

\begin{prooftree}
  \AxiomC{$(\sigma,p,e) \bigstep_E d$}
  \AxiomC{$(\sigma,p,e') \bigstep_E d'$}
  \RightLabel{$d \equiv d'\qquad (\mathit{eq-true})$}
  \BinaryInfC{$(\sigma,p,e\isEq\,e') \bigstep_E \trueD$}
\end{prooftree}

\begin{prooftree}
  \AxiomC{$(\sigma,p,e) \bigstep_E d$}
  \AxiomC{$(\sigma,p,e') \bigstep_E d'$}
  \RightLabel{$d \not\equiv d'\qquad (\mathit{eq-false})$}
  \BinaryInfC{$(\sigma,p,e\,\isEq\,e') \bigstep_E \falseD$}
\end{prooftree}

\begin{prooftree}
  \AxiomC{ }
  \RightLabel{$(\mathit{star})$}
  \UnaryInfC{$(\sigma,p,\star)\bigstep_E \toD{p}$}
\end{prooftree}

En la evaluación de $\univ$ es necesario tener definida la evaluación
de programas.

\begin{prooftree}
  \AxiomC{$(\sigma,p,e) \bigstep_E d$}
  \AxiomC{$(\sigma,p,e') \bigstep_E d'$}
  \AxiomC{$(\sigma,p',d') \bigstep_P d''$}
  \RightLabel{$\fromD{d} = p'\qquad (\mathit{univ})$}
  \TrinaryInfC{$(\sigma,p,\univ\,e\,e') \bigstep_E d''$}
\end{prooftree}

Las configuraciones no terminales para evaluar comandos son tuplas
$(\sigma,p,c)$ de estados, programas y comandos, mientras que las
configuraciones terminales son estados.

  \begin{prooftree}
    \AxiomC{$(\sigma,p,e) \bigstep_E d$} 
    \RightLabel{$(\mathit{assgn})$}
    \UnaryInfC{$(\sigma,p,v\,\assign\,e) \bigstep_C [\sigma\.\mid\.v
      \colon d]$}
  \end{prooftree}

  \begin{prooftree}
    \AxiomC{$(\sigma,p,c) \bigstep_C \sigma'$}
    \AxiomC{$(\sigma',p,c') \bigstep_C \sigma''$}
    \RightLabel{$(\mathit{seq})$}
    \BinaryInfC{$(\sigma,p,c\;;c') \bigstep_C \sigma''$}
  \end{prooftree}


\begin{prooftree}
  \AxiomC{$(\sigma,p,e) \bigstep_E \falseD$}
  \RightLabel{$(\mathit{while-false})$}
  \UnaryInfC{$(\sigma,p,\whilestmt{e}{c}) \bigstep_C \sigma$}  
\end{prooftree}

\begin{prooftree}
  \AxiomC{$(\sigma,p,e) \bigstep_E \trueD$}
  \AxiomC{$(\sigma,p,c) \bigstep_C \sigma'$}
  \AxiomC{$(\sigma',p,\whilestmt{e}{c}) \bigstep_C \sigma''$}
  \RightLabel{$(\mathit{while-true})$}
  \TrinaryInfC{$(\sigma,p,\whilestmt{e}{c}) \bigstep_C \sigma''$}
\end{prooftree}

Finalmente damos las reglas para los programas, ahora las
configuraciones no terminales son triplas $(\sigma,p,d)$ de estados,
programas y valores, mientras que las terminales son valores.

\begin{prooftree}
  \AxiomC{$([\sigma\.\mid\. v\colon d],\readV\,v\,\seq\,c\,\seq\,\writeV\,w,c)\bigstep_C \sigma'$}
  \RightLabel{$(\mathit{prg})$}
  \UnaryInfC{$(\sigma,\readV\,v\,\seq\,c\,\seq\,\writeV\,w,d) \bigstep_P \sigma'\,w$}
\end{prooftree}

\end{document}